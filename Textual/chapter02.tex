\chapter{Motivation and Initial Definitions}
\label{chap2}
\newcommand{\nesetril}{Ne{\v{s}}et{\v{r}}il}
\newcommand{\devos}{DeVos}
\newcommand{\samal}{{\v{S}}{\'a}mal}
\newcommand{\bads}{{\rm Bads  }}
\newcommand{\girth}{g}
\newcommand{\dist}{{\rm dist}}
% \newcommand{\samal}{{\v{S}}{\'a}mal} 
\newcommand{\cleb}{Clebsch graph}

\newcommand{\cost}{{\rm cost}}
%
% \section{TODOS}
% \begin{itemize}
% % \item Testar com o seguinte grafo de Clebsch modificado ao adicionarmos todas as arestas incidentes ao 0.
% % \item Testar nível 4 com quadruplas 
% % \item Remover referência a super grafo 
% % \item Testar todas as quadruplas e calcular números de raizes possíveis 
% % \item Adpatar a parte de replacements para Clebsch novamente e falar dos triviais 
%
% \item Prova de conceito
% \item Na seção de resultados falar que a técnica funciona pro supergrafo de Clebsch. Infelizmente, há um homomorfismo trivial para este grafo.
%
% \item Que condições a gente precisa pra essa estratégia funcionar? (1) que qualquer par de vértices com distância 2 tenha pelo menos dois vizinhos em comum.
%
% \item Perspectiva geral: recebemos um grafo cúbico que admite um homomorfismo a um grafo H, será que conseguimos evitar algum vértice adicionando uma condição na cintura de G?
%
% \end{itemize}

In what follows, if \(m\) is a function \(m\colon A \to B\) and \(A'\subseteq A\),
we denote by \(m|_{A'}\) the restriction of \(m\) to \(A'\),
i.e., the function \(m|_{A'}\colon A' \to B\)
such that \(m|_{A'}(x) = m(x)\) for every \(x\in A'\).

Let \(T\) be a tree in a cubic graph \(G\).
A vertex \(x\) such that \(d_T(x) = 1\) is called a \emph{leaf} of \(T\),
otherwise \(x\) is called \emph{internal}.
We say that \(T\) is \emph{cubic} if \(d_T(x)\in\{1,3\}\) for every vertex of \(T\).

%TODO talvez queiramos isso para sequências de vértices.
%TODO verificar se usamos apenas o conjunto de cores
% Let $h$ be any homomorphism $h \colon G \to G'$ and $T \subseteq G$ a tree, we define $h(L(T))$ as the set $\{ h(x) : x \in L(T)\}$.

Let $h$ be a homomorphism $h \colon G \to G'$ and let \(S\) be a tuple \((u_1,\ldots,u_s)\) of vertices of a graph.
Denote the tuple \(\big(h(u_1),\ldots, h(u_s)\big)\) as $h(s)$.

Given a complete tree $T$ of height \(h\geq 1\), the \emph{internal induced subtree} $T^{(1)}$ is the subtree of \(T\) induced by the vertices in $V(T) \setminus L(T)$. 
Additionally, we define $T^{(2)}$ to be $(T^{(1)})^{(1)}$ and, 
more generally, if \(d < h\), then $T^{(d)} = (T^{(d-1)})^{(1)}$.


\subsection{The Clebsch graph}\label{sec:clebsch}
The Clebsch graph (see Figure~\ref{fig:clebsch_graph}) is the $5$-regular graph \(CH = (V_1 \cup V_2 \cup V_3, E)\), where
$V_1 = \big\{ \{x\} : \text{ for all } x \in [5] \big\}$, $V_2= \big\{ \{x,y\} : \text{ for all } x, y \in [5] \text{ such that } x < y \big\}$, $V_3 = \big\{ \{ 1,2,3,4,5 \} \big\}$ and
\[E = \big\{ XY : X \in V_1,\ Y \in V_2\cup V_3\text{ and } X \subseteq Y\big\} \cup \big\{ XY : X,\ Y \in V_2,\ X \cap Y = \emptyset  \big\}.\]

Throughout the text we refer to the vertices of \(\ch\) as $0, \ldots, 15$ as show in Figure~\ref{fig:clebsch_graph} 
and, to avoid confusion with the vertices of the studied cubic graphs, we refer to the vertices of \(\ch\) as \emph{colors}.
Moreover, we often refer to homomorphims to $\ch$ as colorings.
A key observation with respect to \(\ch\) that is used throughout the text is that every pair of (not necessarily distinct) colors has either zero, two, or five common neighbors.
\begin{fact}\label{fact:common_neighbors}
    If \(u,v \in V(\ch)\), then \(|N(u)\cap N(v)| \in \{0,2,5\}\).
\end{fact}

\section{The Pentagon Problem}
\label{sec:the-pentagon-problem}

This work concerns about the \emph{\nesetril{} Pentagon Problem} stated below. 

% já definiu homorphism?
% já definiu o C5?
% Já definiu cubic graph?
% já definiu cintura?
\begin{conjecture}\label{conj:pentagon-problem}
    There is a constant \(g_0\in \mathbb{N}\) for which the following holds.
    If \(G\) is a cubic graph with girth at least \(g_0\),
    then \(G\) has a homomorphism to $C_5$.
\end{conjecture}

Conjecture~\ref{conj:pentagon-problem} was originally proposed in \nesetril{} problem sessions~(see \cite{nesetril1999aspects}). 
By Brook's Theorem (see~\cite{Brooks1941} or~\cite[Theorem 5.2.4]{diestel2017graph}),
every triangle-free cubic graph has chromatic number at most \(3\),
and hence is homomorphic to \(C_3\).
On the other hand, \cite{HATAMI2005319} showed that Conjecture~\ref{conj:pentagon-problem} does not hold 
if we replace \(C_5\) by \(C_\ell\) for any \(\ell>5\),
i.e., that for every \(\ell\) and \(g\) there is a cubic graph of girth at least \(g\) that is not homomorphic to \(C_\ell\).
Conjecture~\ref{conj:pentagon-problem} deals then with the remaining case of \(C_5\).

\begin{figure}[h]
     \begin{subfigure}[b]{0.45\textwidth}
         \centering
         \input{Imagens/C5}
         \caption{The cycle $C_5$ of length $5$}
         \label{fig:c5}
     \end{subfigure}
     \begin{subfigure}[b]{0.45\textwidth}
         \centering
         \input{Imagens/clebsch_graph}
         \caption{The Clebsch Graph $\ch$}
         \label{fig:clebsch_graph}
     \end{subfigure}
     % \caption{Left: the cycle $C_5$ of length $5$; Right: The Clebsch Graph $\ch$}
     \caption{}
     \label{}
\end{figure}

\newcommand{\ch}{{CH}} 
\newcommand{\chzero}{{CH_0}}
\newcommand{\chone}{{CH_1}}
As an approximate result, in a computer-assisted proof, \cite{devos2011high} showed that every graph $G$ with $\Delta(G) \leq 3 $ and girth at least 17 has a homomorphism to the \emph{Clebsch graph}
that can be computed by a liner time algorithm.

\begin{theorem}[\devos--\samal]\label{thm:DevosSamal11}
Every graph of maximum degree 3 and girth at least 17 is homomorphic to $\ch$.
\end{theorem}

%TODO: falar que "It is then natural to consider other computer search methods in order to improve Theorem 2"

% TODO falar em algum lugar da transitividade dos homomorfismos: se G -> H e H-> R, então G -> R

% Observe ao removermos os vértices \(0,1,2,7,12\) do grafo de Clebsch, obtemos um grafo que admite um homomorfismo para o \(C_5\).
% Assim, se há um homomorfismo de \(G\) ao grafo de Clebsch que evita tais vértices, há um homomorfismo de G para C_5.
% Em particular, sejam V1,...,V5 as classes de um homomorfismo h^-: C^- = Clebsch-{0,1,2,7,12} -> C5, tais que os vértices de V_i são adjacentes apenas aos vértices de V_{i-1} e V_{i+1}, onde a soma nos índices é tomada módulo 5.
% 
We consider the possibility of building upon Theorem~\ref{thm:DevosSamal11} to get a stronger result towards Conjecture~\ref{conj:pentagon-problem} using the following simple observation:
there is a small set \(B\subseteq V(\ch)\) such that the graph obtained from \(\ch\) by removing the vertices in \(B\)
is homomorphic to \(C_5\).
Although there are several sets with the property above, in this work we use the set \(B_0 = \{0,1,2,7,12\}\),
which we call the set of \emph{undesired colors},
and we denote by \(\ch_0\) the graph obtained from \(\ch\) by removing the vertices in \(B_0\).
It's not hard to check that \(CH_0\) has a homomorphism \(h_0\) to $C_5$\footnote{For example pick the graph classes \{4, 8\}, \{9,10,14\}, \{6,11,15\}, \{3, 13\}, \{5\}},
and hence, to solve Conjecture~\ref{conj:pentagon-problem}, it's enough to prove that there is a constant \(g_0\) for which every cubic graph with girth at least \(g_0\) has a homomorphism to \(CH_0\),
i.e., a homomorphism to \(CH\) that avoids the vertices in \(B_0\).
%
The objective of this work is to explore computer assisted methods to 
find a partial result in this direction.
More specifically we want to find a minimum girth \(g_0\)
for which all cubic graphs of girth at least $g_0$ have a homomorphism to 
the graph \(\chone\) obtained from \(\ch\) by removing a non-empty subset $B_1 \subseteq B_0$. 


We establish a theoretical framework for locally recoloring the extended neighborhood of a vertex, that, due to the high girth condition, is isomorphic to a tree, 
in order to avoid undesired colors.
We use this framework to prove the trivial result that we can always remove one of the undesired vertices from any homomorphism from a cubic graph of girth at least $17$ to a degenerate form of Clebsch graph that contains a few triangles. Furthermore, we try unsuccessfully to use it to remove a color from any homomorphism from a cubic graph of girth at least $g$ to the Clebsch graph,
and present some statistical results regarding the number of \emph{configurations} (see Section~\ref{sec:configurations})
that can be ``improved'' using this method.
On the other hand, we present an infinite family of configurations that, independently of the girth, cannot be locally improved (see Chapter~\ref{chap5}).
This proves the non-effectiveness of this method.

The main contribution of this work is the theoretical foundation that enables this framework which
can be further used to explore different subgraphs (See Chapter~\ref{chap6})
as cycles, for which it is still not clear if the above intrinsically bad configurations also provide obstructions.

\section{Theoretical foundation}\label{sec:theoretical-foundation}

In order to avoid undesired colors, we need to 

Let \(G\) be a subcubic graph, and let \(h\) be a homomorphism $G \to \CH$.
We define the \emph{cost} of $h$ as 
\[\cost(h)~=~|\{x : \text{ for all } x \in V(G) \text{ and } h(x) \in B_1 \}|.\] 

Observe that, if \(\cost(h) = 0\), then \(h\) avoids the vertices in \(B_1\),
and hence, as observed in Section~\ref{sec:the-pentagon-problem},
\(h\) is a homomorphism from $G$ to $\chone$.


Now, let \(G\) be a cubic graph,
let \(T\subseteq G\) be a cubic tree,
and suppose that exists a homomorphism \(h \colon G \to \ch\).
We say that a homomorphism \(h' \colon T \to \ch\) is \(h\)-\emph{compatible} if \(h'|_{L(T)} = h|_{L(T)}\),
i.e., if \(h'(x) = h(x)\) for every \(x\in L(T)\),
and we define the \((h,h')\)-\emph{modified} map $h^* \colon G \to \ch$ by
\[ 
h^*(v) = 
     \begin{cases}
       h'(v) &\quad\text{if } v\in V(T)\\
       h(v) &\quad\text{if } v\notin V(T).
     \end{cases}
\]

By the definition of \(h\)-compatible, 
if \(d_T(u) = d_G(u)\) for every vertex of \(T\) that is not a leaf, 
then $h^*$ is a homomorphism from \(G\) to \(CH\).
Indeed, let \(uv\in E(G)\). 
If \(uv\in E(T)\), then \(h'(u)h'(v) \in E(CH)\) because \(h'\) is a homomorphism from \(T\) to \(CH\).
Now, suppose that \(uv\notin E(T)\).
If both $u$ and $v$ are in $V(G) \setminus V(T)$ we have $h^*(u)h^*(v) = h(u)h(v)$ which in $E(CH)$ because \(h\) is is a homomorphism from \(G\) to \(CH\).
If $u \in V(T)$ and $v \in V(G) \setminus V(T)$, then $u$ is a leaf of $T$.
By the definition of \(h\)-compatible we have $h'(u) = h(u)$ and hence $h^*(u)h^*(v) = h(u)h(v)$ which is in \(E(CH)\) because \(h\) is is a homomorphism from \(G\) to \(CH\).

%Definição alternativa de custo
%Now, let \(G'\) be a subgraph of \(G\), 
%and let \(m\colon V(G')\to V(H)\).
%We define the \(\cost(m)\) as the number of elements of \(V(G')\) mapped to $0$,
%i.e., the size of \(\{x\in V(G') : m(x) = 0\}\).
The following lemma comes naturally.

\begin{lemma}\label{lemma:cost-improvement}
    Let \(G\) be a cubic graph for which there is a homomorphism \(h\) from \(G\) to \(CH\), let $T \subseteq G$ a cubic tree, $h'$ a $h$-compatible homomorphism from $T$ to $\ch$,
    and let \(h^*\) be the \((h,h')\)-modified homomorphism,
    then \[\cost(h^*) = \cost(h) - \cost(h|_T) + \cost(h').\]
\end{lemma}

% Similarly, if both $u$ and $v$ are in $G \setminus T$ we have $(h_1(u), h_1(v)) = (c(u), c(v)) $ which by definition is in $E(H)$. If $u \in T$ and $v \in G \setminus T$ we have that necessarily u is a leaf of $T$ and therefore, by the definition of \(c\)-\(\compatible\), $ c_2(u) = c(u)$ and $(h_1(u), h_1(v)) = (c(u), c(v)) $ which by definition is in $E(H)$. 
% We also have that : 
%%%USAR ALIGN AQUI
% $$
% cost(h_1) = | \{x\in V(G) : h_1(x)\in\bads\} | 
% $$
% \[= | \{x\in V(G \setminus T) : c(x)\in\bads\} | + | \{x\in V(T) : c_2(x)\in\bads\} | \]
% \[= | \{x\in V(G) : c(x)\in\bads\} | -  |\{x\in V(T) : c(x)\in\bads\} | + cost(c_2) \]
% \[= cost(c) - | \{x\in V( T) : c(x)\in\bads\} | + cost(c_2) \]
% \[= cost(c) - cost(c|_T) + cost(c_2) \]
\begin{proof}
    \begin{align*}
    \cost(h^*) 
        & = | \{x\in V(G) : h^*(x)\in\bads\} | \\
        & = | \{x\in V(G) \setminus V(T)) : h(x)\in\bads\} | + | \{x\in V(T) : h'(x)\in\bads\} | \\
        & = | \{x\in V(G) : h(x)\in\bads\} | -  |\{x\in V(T) : h(x)\in\bads\} | + cost(h') \\
        & = \cost(h) - | \{x\in V( T) : h(x)\in\bads\} | + cost(h') \\
        & = \cost(h) - \cost(h|_T) + \cost(h') \qedhere.
    \end{align*}
\end{proof}

Now, suppose that \(h\) is a homomorphism from \(G\) to \(\ch\) 
that minimizes \(\cost(h)\).
Let \(T\subseteq G\) be a cubic tree,
and let \(h'\) be a \(h\)-compatible homomorphism of \(T\).
%
Consider the \((h,h')\)-modified homomorphism \(h^*\).
If \(\cost(h') < \cost(h|_T)\), then \(h^*\)
is a homomorphism of \(G\) in \(\CH\) for which, by Lemma~\ref{lemma:cost-improvement}, we have \[\cost(h^*) = \cost(h) - \cost(h|_T) + \cost(h') < \cost(h),\]
a contradiction to the minimality of \(h\).
Therefore, for every cubic tree \(T\subseteq G\) there is no \(h\)-compatible homomorphism \(h'\) of \(T\) for which \(\cost(h') < \cost(h|_T)\).
This yields the following lemma

\begin{lemma}\label{lemma:there-is-no-improvement}
Let \(G\) be a cubic graph with girth at least \(\girth{}\),
and let \(h\) be a homomorphism from \(G\) to \(CH\) 
that minimizes \(\cost(h)\).
Now, let \(T\subseteq G\) be a cubic tree. 
Then there is no \(h\)-compatible homomorphism \(h'\colon T \to CH\)
such that \(\cost(h') < \cost(h|_T)\).
\end{lemma}

The methods developed in this work mean to explore the following conjecture.

\begin{conjecture}\label{conj:main-conjecture}
Let \(G\) be a cubic graph with girth at least \(\girth{}\) for which there is a homomorphism \(h\) from \(G\) to \(CH\).
If \(\cost(h) > 0\),
then there is a cubic tree \(T\subseteq G\)
and a \(h\)-compatible homomorphism \(h'\) from \(T\) to \(CH\)
such that \(\cost(h') < \cost(h|_T)\).
\end{conjecture}

If Conjecture~\ref{conj:main-conjecture} holds,
then if \(h\) is a homomorphism from \(G\) to \(CH\) 
that minimizes \(\cost(h)\),
then \(\cost(h) = 0\).
Therefore, as a natural consequence of Lemma~\ref{lemma:cost-improvement} and Conjecture~\ref{conj:main-conjecture}, we have the following.

%TODO: revisar isso quando checar se a prova funciona para H no lugar de H^+
\begin{theorem}\label{thm:main}
    % Suppose that Conjecture~\ref{conj:main-conjecture} holds.
    % If \(G\) is a cubic graph of girth at least \(\girth\),
    % then there is a homomorphism from $G\to CH_1$.
% 
    If Conjecture~\ref{conj:main-conjecture} holds,
    then every cubic graphs with girth at least \(\girth\) has a homomorphism to \(\chone\).
\end{theorem}

\begin{proof}
Let \(G\) be as in the statement.
By Theorem~\ref{thm:DevosSamal11}, there is a homomorphism from \(G\) to \(CH\).
Let \(h\) be a homomorphism from \(G\) to \(CH\) that minimizes \(\cost(h)\).
Suppose \(\cost(h) > 0\). 
If Conjecture~\ref{conj:main-conjecture} holds, then there is a cubic tree \(T\subseteq G\)
and a \(h\)-compatible homomorphism \(h'\) from \(T\) to \(CH\)
such that \(\cost(h') < \cost(h|_T)\).
Now, let \(h^*\) be the \((h,h')\)-modified homomorphism from \(G\) to \(CH\).
By Lemma~\ref{lemma:cost-improvement}, we have \(\cost(h^*) = \cost(h) - \cost(h|_T) + \cost(h') < \cost(h)\),
a contradiction to the choice of \(h\).
Therefore, we have \(\cost(h) = 0\), and hence \(h\) is a homomorphism from \(G\) to \(\chone\) as desired.
\end{proof}


