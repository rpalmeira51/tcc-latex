\chapter{Disproving the conjuncture}\label{chap5}
In this chapter, 
we show that the Algorithmic Approach is unable to produce the desired results to proof Conjecture~\ref{conj:main-conjecture} 
as Algorithm~\ref{alg:naive-conjecture-holds} (\texttt{GenerateUnsolvableConfs}) does not return zero regardless of the complete cubic tree passed as argument.
We do this by introducing a process that generates a bad configuration for any complete cubic tree, regardless of its size.
Therefore the negative results presented on Chapter~\ref{chap5} are not the consequence of insufficient computing power, 
but a natural consequence of the properties of configurations.

% Let $c(v)$ be the child vertices of $v$ 
% $h(c(v))$ be $\{ h(c) | c \in c(v) } \}$
% $p(u)$ be the \emph{possible colors} that $u$ can be colored with in respect to it's children.
%
\begin{lemma}
Let $T_b$ be a complete binary tree of height at least two, 
$conf$ a configuration of $T_b$, 
$u$ and $v$ two siblings vertices of $L(T_b^{(-1)})$, 
$\{w, x\}$ and $\{y, z\}$ the children of $u$ and $v$, respectively; 
and $IG$ be the subgraph of $\ch$ induced by $(N_{conf(w)}~\cap~N_{conf(x)})~\cup~(N_{conf(y)}~\cap~N_{conf(z)}))$.

If $IG$ is a $P_4$, then there is only one possible color $c_u$ for $u$ and $c_v$ for $v$ such that $c_u$ and $c_v$ have common neighbors 
and, as consequence, there is no feasible configuration $conf^{(-1)}$ of $T_b^{(-1)}$ such that $conf^{(-1)}(u) \neq c_u$ or $conf^{(-1)}(v) \neq c_v$.
\end{lemma}
\begin{proof}
Suppose the contrary is true, i.e, $IG$ is a $P_4$, but the statement does not hold.
First, note that we have only two options $c_1$ and $c_2$ to color $u$ and two options $c_3$ and $c_4$ for $v$. 
This can be easily deduced from Fact~\ref{fact:common_neighbors} and the observation that $|V(P_4)| = 4$.
Also, that $c_1, c_2 \notin E(\ch)$ and $c_3, c_4 \notin E(\ch)$, because otherwise there would be triangles in $\ch$.
Therefore all 3 edges of $IG$ are between $\{c_1,c_2\}$ and $\{c_3,c_4\}$, 
thus, there is only one non edge $\{c_u,c_v\}$ between these partitions with $c_u \in \{c_1,c_2\}$ and $c_v \in \{c_3,c_4\}$.
It is easy to see that $c_u$ and $c_v$ is a coloring for $u$ and $v$ and that $c_u$ and $c_v$ have common neighbors.
Any other coloring of $u$ and $v$ with the same property would imply triangles on $\ch$, a contradiction.
\end{proof}

\begin{lemma}\label{lemma:pair-bad-coloring}
   For every pair $\{c_u, c_v\}$ of non adjacent vertices in $\ch$, 
   there are vertices $c_1, c_2$ adjacent to $c_u$ and 
   vertices $c_3, c_4$ adjacent to $c_v$ 
   such that $\{c_u, c_v\}$ is the only non edge between $N_{c_1}~\cap~N_{c_2}$ and $N_{c_3}~\cap~N_{c_4}$ 
   on the induced $P4$ subgraph by $(N_{c_1}~\cap~N_{c_2})~\cup~(N_{c_3}~\cap~N_{c_4})$.
\end{lemma}
\begin{proof}
We prove this by introducing a process that is always able to produce $c_1$, $c_2$, $c_3$ and $c_4$ such that the condition holds.
Chose any $c_u$ and $c_v$ distinct and non adjacent.
By Fact~\ref{fact:common_neighbors}, they have two common neighbors and tree distinct ones. 
Without loss of generality, 
choose $c_w$ to be one of the neighbors of $c_v$ which is not a neighbor of $c_u$ and 
let $c_1, c_2$ be the two common neighbors of $c_u$ and $c_w$.
Note that both $c_1$ and $c_2$ are not neighbors of $c_v$ as that would imply triangles.
Then let $c_3, c_4$ be the common neighbors of $c_v$ and $c_1$, 
we know that subgraph induced by $(N_{c_1}~\cap~N_{c_2})~\cup~(N_{c_3}~\cap~N_{c_4})$ is 
the subgraph induced by $\{c_u, c_v, c_w , c_1\}$ which is a $P4$ with $\{c_u, c_v\}$, $\{c_u, c_w\}$, $\{c_v, c_1\}$ as non edges, 
the first being the only between the partitions.
\end{proof}

The following corollary follows naturally.
\begin{corollary}\label{coro:every-parents-configuration}
For every feasible configuration $conf_b$ of a binary tree $T_b^{(-1)}$ with height greater than one, 
if $conf_b$ colors sibling vertices distinctly,
we can construct a configuration of $T_b$ such that its only parents configuration is $conf_b$.
\end{corollary}

\begin{lemma}\label{lemma:root-coloring}
   Let $T_b$ be a binary tree rooted in $r$ with height $l$ greater than zero and $c_1$ and $c_2$ be two non adjacent vertices of $\ch$, 
   there is always a configuration of $T_b$ for which there are only two completions $h_1$ and $h_2$ 
   with $h_1(v) = h_2(v)$ for every $v$ in $V(T_b) \setminus \{r\}$, $h_1(r) = c_1$ and $h_2(r) = c_2$.
\end{lemma}
\begin{proof}
   Lets proof by induction.
   In the base case, 
   $T_b$ has height one ($l$ = 1) with $r$ as root and vertices $v_1$ and $v_2$ as children. 
   Then, it is enough to color $v_1$ and $v_2$ with the common neighbors of $c_1$ and $c_2$.
   In the inductive step,
   we know that the condition holds for $l-1$, i.e, $T_b^{(-1)}$ has a configuration $conf_b^{(-1)}$
   for which there are only two completions $h_1$ and $h_2$ such that $h_1(v) = h_2(v)$ for every $v$ in $V(T_b^{(-1)}) \setminus \{r\}$, $h_1(r) = c_1$ and $h_2(r) = c_2$.

   Then, by Corollary~\ref{coro:every-parents-configuration}, 
   there is a feasible configuration $conf_b$ for which $conf_b^{(-1)}$ is the only parents configuration, 
   therefore any completion of $conf_b$ must color $L(T^{(-1)})$ with $conf_b^{(-1)}$ and, 
   as consequence, only two completions exist for $conf_b$. 
\end{proof}

\begin{lemma}\label{lemma:always-bad}
   Given any complete cubic tree $T$ of arbitrary large size $l > 1$ there is at least one bad configuration of $T$.
\end{lemma}
\begin{proof}
   Let $r$ be the root of $T$ and $u$, $v$ and $w$ the children of $r$.
   Without loss of generality, chose any $b \in B_1$ 
   and let $T_b$ be the complete binary tree obtained by removing $u$ and all vertices for which the shortest path to $r$ includes $u$.
   Let $c_v$ and $c_w$ be the common neighbors of $b$ and any other color $c_1$ and fix $conf_b$ to be a configuration for which all completions 
   color $v$ with $c_v$ and $w$ with $c_w$~(See~Lemma~\ref{lemma:root-coloring}).
   Note that $b$ has five neighbors, two in common with $c_1$ and tree that are neighbors of $b$, but not of $c_1$.
   Let $c_2$ and $c_3$ be two from the latter 
   and $T_{b_2} \subset T$ be the binary subtree rooted in $u$ such that $T_{b_2} \cup T_b = T$.
   We know there is a configuration $conf_{b_2}$ such that $u$ can only be colored with $c_2$ or $c_3$ (See Lemma~\ref{lemma:root-coloring}). 
   Note that the intersection of the neighborhood of $c_2$, $c_v$ and $c_w$ is $b$ and the same is true for $c_3$, $c_v$ and $c_w$.
   Therefore we have that $r$ can only be colored with $b$ in any completion, 
   and, as consequence, the configuration $conf$ of $T$ formed by coloring every leaf of $T$ in $T_b$ with $conf_b$ 
   and every leaf in $T_{b_2}$ with $conf_{b_2}$ is bad.
\end{proof}

Lemma~\ref{lemma:always-bad} shows that for any arbitrary large complete cubic tree, there always will be at least one bad configuration.
One can note that Algorithm~\ref{alg:naive-conjecture-holds} (\texttt{GenerateUnsolvableConfs}) fails to produce the desired results 
if there are any \emph{unsolvable} bad configurations.
However it follows from the definition that if a configuration of $T$ is solvable 
there is good configuration $conf^+$ on a super tree $T^+ \supset T$ such that there is one completion $h$ for witch $h_{|L(T)} = conf$.
Therefore Lemma~\ref{lemma:always-bad} also holds for unsolvable configurations \emph{mutatis mutandis}.


Consegui usar =).

