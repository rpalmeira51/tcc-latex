\chapter{Disproving the conjuncture}



%
%- Assume árvore binária completa  
%- A gente pega dois vértices irmãos quaisquer
%- Considera os 4 filhos deles e agora considera as cores possiveos para recoloris esses dois vértices 
%
%- Seja c^(-1) uma configuração de T^(-1), então há configuração c de T cuja única compleção dos pais de L(T) é c^(-1)
%
%- indução: se T tem altura 1. Então c^(-1) consiste apenas de uma cor. Seja a,b,c três vizinhos distintos de c^(-1) no Grafo de Clebsch. 
%Suponha que T tem altura h, e seja c^(-1) = (f_1,...,f_k) uma coloração de L(T^(-1)).
%Para cada f_i 
%Para cada par u,v de irmãos em L(T^(-1)), sejam a,b, e c,d os filhos de u e v tal que os pares (a,b) e (c,d)   e c(a),c(u) $\in E(CH)$ ... 
%observe que c(u) e c(v) são cores não adjacentes em CLEBSCH. 
%Seja c(u') (resp. c(v')) uma cor não adjacente a c(u) (resp. c(v)) diferente de c(v) (resp. c(u))
%tais que c(u') e c(v') são vizinhos. 
%*** Seja c(u') e c(v') cores tais que (c(u), c(v'), c(u'), c(v)) induzem um P4 induzido em CLEBSCH.
%Argumentar que isso é sempre possível.
%Agora seja c(a) e c(b) (resp. c(c) e c(d)) os vizinhos em comum de c(u) e c(u') (resp. c(v) e c(v')).
%Colora os filhos a,b de u com c(a) e c(b), e os filhos c,d de v com c(c) e c(d).
%Isso nos dá uma configuração para T.
%Observe que as únicas cores possíveis para u (reps. v) são c(u) e c(u') (resp. c(v) e c(v')). 
%Como  (c(u), c(v'), c(u'), c(v)) induzem um P4 induzido em CLEBSCH,
%e a cor c1 de u (resp. c2 de v) deve ser uma cor em \(\{c(u),c(u')\}\) (resp. \(\{c(v),c(v')\}\) tal que c1 e c2 não são adjacentes, então temos que c1 = c(u) e c2 = c(v). 


For any pair $p$ of distinct non adjacent vertices of the Clebsch Graph, there are two and only two common neighbours of both vertices in $p$.  

We shall call the \emph{extension} of a pair $p$, the induced subgraph formed by the vertices in $p$ and it's common neighbours.

Now, let $p_1$ and $p_2$ be two distinct pairs of distinct non adjacent vertices of the Clebsch Graph and consider the induced subgraph $S$ formed by the common neighbors $u$ and $v$ of $p_1$ and $z$ and $w$ of $p_2$. 


Let $o$ be a children ordering and suppose, without loss of generality, that $p_1$ is smaller than $p_2$ in this order and also that both pairs are ordered.

Then let $T$ be a complete binary tree of height 2 rooted in $r$ and $conf$ be a any configuration in $T$ such that it's signature is the sequence formed by the two vertices in $p_1$ followed by the two in $p_2$. 

We know that any $conf^{p}$ must color the first vertex with $u$ or $z$ and the second one with $z$ or $w$. 
Note that $S$ is bipartite graph with $\{u,v\}$ and $\{z, w\}$ as the independent components.
Then is easy to see that, if $conf$ is valid, there must be at least one non edge between the two partition such that the vertices of this non edge have common neighbours. 
If there is precisely one non edge, the root can only be colored with the two colors adjacent to both vertices of the non edge for any completion of $conf$. 

Given any pair of non adjacent vertices in $CH$, we can always construct a configuration $conf$ in $T$ such that all of its completions color $r$ with one of the pair's vertices. 



We can always produce a configuration $conf$ such that completion of $conf$ can only color the root with any two non adjacent vertices. 

Now, 





Let's recall the conjucture as previously stated: 
% \begin{conjecture}
% Let \(G\) be a cubic graph with girth at least \(\girth{}\) for which there is an homomorphism \(h\) from \(G\) to \(CH\).
% If \(\cost(h) > 0\),
% then there is a cubic tree \(T\subseteq G\)
% and a \(h\)-compatible homomorphism \(h'\) from \(T\) to \(CH\)
% such that \(\cost(h') < \cost(h|_T)\).
% \end{conjecture}



First, let's assume that the tree $T$ is balanced.