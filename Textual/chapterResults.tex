\chapter{Experimental Results} \label{chap:chap5}

O quero falar: 
Implemtacao em cpp e sagemath.
Resultados do Problema
Diferenca de performance 
Subproblema 
Resultado do subproblema 
Modificado e nao modificado 


PERGUNTA FABIO: TENHO QUE EXPLICAR O QUE É C++?


In this chapter, we show the results obtained by coding and executing computer programs which implemented the algorithms introduced in Chapter~\ref{chap:alg-approach}, Chapter~\ref{chap:optimizations} and Appendix~\ref{apx:opt-algs} in C++ and, later, we also developed a second implementation and checker on SageMath.
We chose C++ for the initial implementation due to the asymptotic complexity of the algorithms (see Section~\ref{sec:complexity-analysys}), 
as it makes performance critical to be able to classify configurations of non-negligible size 
and 
C++ is a system programming language, hence, has no runtime overhead and it is ideal for performance sensitive implementations.
Alternatives with the same characteristics were mainly C and Rust. 
We didn't chose C for the lack of useful high-level constructs like vector (specifically std::pmr::vector)\footnote{See \url{https://en.cppreference.com/w/cpp/memory/polymorphic_allocator.html} for more information} that are available on C++ standard library and were widely used on our implementation.
Rust were discarded for a lack of familiarity.

%TODO: Definir solvability table
Recall that the main algorithm (See~Algorithm~\ref{alg:opt-cojecture-holds}) can be split on the following steps: 
1) Classify bad and good configurations of a complete cubic tree of a predefined size;
2) Construct the solvability table for the bad configurations and check if the there are replacements such that all unsolvable bad configurations on the solvability table can be solved.
Step 2 be simplified if minimal replacements (See~Definition~\ref{def:minimal-replacements}) were previously computed, we will label this process as step 0).
In this case, for each pair $(c_0, C_s)$ on the solvability table we only need to check if $C_s$ is a superset of any minimal replacement $(c_0, C)$.
This combined with the time needed to execute the steps for non trivial constants made us, during the development and testing of the C++ implementation, to devise a file oriented approach
such that, on the end of each step, the result of the computation is written to files.

This allowed steps to be computed in a temporally decoupled way, as the program is able to read the necessary information from the previously steps files and continue accordingly.
Another benefit we got with this approach was to be able to develop a checker for the file outputs in SageMath.
SageMath, according to the official website, is "a free open-source mathematics software system licensed under the GPL. It builds on top of many existing open-source packages: NumPy, SciPy, matplotlib, Sympy, Maxima, GAP, FLINT, R and many more.",
and, most importantly for this work, it has a extensive undirect graph module that provides a lot of built in functionality in a python like syntax which is easier to implement and reason about,
since it hides complexities related to memory management and low level operations which are an essential part of system programming languages.
This allowed us to more easily recheck the results produced by the C++ implementation. 
In the end, we rewrote the all algorithms in SageMath, as it the checkers already demanded most of them 
and
was useful to compare the timings of each implementations (see Section~\ref{sec:time-measurements}).
The algorithms we presented in pseudo code on Chapter~\ref{chap:optimizations} and Appendix~\ref{apx:opt-algs} are the ones used in our SageMath implementation 
slightly modified for brevity and clarity made possible by pseudo code expressiveness.


\section{Time Measurements}\label{sec:time-measurements}
In order to confirm our assumptions about the speed of execution between ours SageMath and C++ implementations 
and to be able to discuss the trade-off between computing speed and prototyping/hypothesis testing speed (See section~\ref{sec:multi-implementations}),
we will show, in every results section, the measurements of the time needed to compute them on each implementation.
All the measurements were done using the tool \texttt{hyperfine}\footnote{The documentation and source code is freely available on \url{https://github.com/sharkdp/hyperfine}}
with 5 warm up runs and 100 measurements runs in a machine with specifications shown on Table~\ref{tab:specs}.
% hyperfine ./bin/program --warmup 5 --runs 100

\begin{table}[h!] \label{tab:specs}
\centering
\caption{}
\begin{tabular}{ll}
\hline
\textbf{Component} & \textbf{Specification} \\
\hline
Processor          & AMD Ryzen 7 5700\\
RAM                & 32 GB DDR4 2400 MHz \\
GPU                & AMD Radeon RX 6600\\
Operating System   & OS: Arch Linux x86_64\\
Kernel Version     & 6.15.7-arch1-1\\
\hline
\end{tabular}
\end{table}

\begin{table}[h!]
\centering
\caption{Execution Time for Different Implementations}
\begin{tabular}{|l|c|c|}
\hline
\textbf{Experiment} & \multicolumn{2}{c|}{\textbf{Execution Time} (ms)} \\
\hline
& C++ & SageMath \\
\hline
Classifying Good and Bads & 23.3 \pm 0.5 & 1450 \pm 45 \\
Exp 2               & 98.7  & 102.3 \\
Exp 3               & 210.1 & 205.4 \\
Exp 4               & 176.3 & 180.0 \\
\hline
\end{tabular}
\label{tab:execution_time}
\end{table}



All of the results presented are considering $B_1 = \{0\}$.

\section{Minimal Replacements}\label{sec:minimal-replacements}

\section{Results for \cleb}
\subsection{Classifying good and bad}
Recall that the algorithms can be resumed as follows (See chapters Chapter~\ref{chap:alg-approach}~and~\ref{chap:optimizations} for details).
On the start, we have one bad configuration $\{0\}$ and fifteen good ones.
For each bad configuration, we produce the children configurations and check which of these are bad or good.
Then we a have a new set of bad configurations and repeat the last step until the configurations are of a tree with height greater then a predefined maximum.
Lastly we check which of the remaining bad configurations are solvable.

On Table~\ref{tab:problem-results} we show the classification of the configurations obtained at each level of the tree.
Its is important to keep 
In this chapter we introduce the results we obtained by 
\begin{table}[h!]\label{tab:problem-results}
\centering
\begin{tabular}{|c|c|c|c|}
\hline
\textbf{Tree Height} & \textbf{Bad Colorings} & \textbf{Good Colorings} & \textbf{Bad Colorings \%} \\
\hline
0 & 1 & 15 & 4\% \\
1 & 10 & 25 & 28\% \\
2 & 1800 & 15501 & 10.40\% \\
\hline
\end{tabular}
\caption{Number of bad and good colorings obtained at each level}
\end{table}

\section{Results for Degenerated}

\section{C++ Worth the troubles?} \label{sec:system-lang-implementation}
